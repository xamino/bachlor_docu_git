\section{Introduction}

This paper represents the initial design and development work for an Android framework to simplify the detection and usage of makers in augmented reality applications.
The framework shall henceforth be named Imagine to differentiate it from other similar frameworks and the libraries it utilizes.

Imagine will initially only target applications written for Android\cite{android} using Java.
It will utilize the OpenCV for Android\cite{opencvandroid} library, a sub-project of the original OpenCV framework\cite{opencv} specifically targeted for Android devices.
By separating the detection and the rendering modules within Imagine, it should be relatively easy to extend the functionality of Imagine to other platforms and other technologies.

To develop and test the framework, we will also provide an application that implements Imagine as the basic feature.
The application will allow for the real-time viewing of virtual objects on top of a marker.
By moving the device, the object can be viewed from different perspectives.
This shall prove the capability and initial concept of the framework and also serve as a starting point for any future derivative work.

This paper consists of all the work done around the actual implementation.
We will look at the basic requirements of the framework and application.
Basic necessities will be listed and reviewed.
We'll also present mockups of the final application and a proposal for the structure of the framework.
Difficulties that arose during development will also be documented, along with possible solutions and commentary.
At the end we will also shortly compare Imagine to other frameworks to put it into perspective.

Apart from the programmatic development of the framework, the implementing application, and the final review of our work, we will also deliver at least basic documentation for the framework.
For the framework this will include a Javadoc\cite{docjava} file for the completed code and a basic tutorial for usage.
The implementing application should be self explanatory, although care will be taken to ensure a low learning hurdle for using it.

\section{Project Context}

This paper is the bachelor thesis of Tamino Hartmann, written at the Faculty of Engineering and Computer Science\cite{faculty} at the Institute of Databases and Information Systems at the University of Ulm\cite{ulmuni}, Germany.
The work commissioned is to create a framework for fast and easy integration of marker-based tracking for possible future projects within the institute, thus decreasing repetitive re-implementation of the same features and allowing a faster development time for implementing applications.
The supervisor was Marc Schickler and the examiner was Professor Doctor Manfred Reichert.
