\section{Introduction}

As mobile devices become more and more powerful and ubiquitous, developers continually realize new uses for them.
One of the fields that is continually reinventing itself is the field of Augmented Reality.
It is the integration of information onto and into our perception of the real world – an augmentation of our reality.
To achieve this many different methods exist.
Many such applications use the user's global position to overlay points of interest onto a camera feed; others use computer vision to track real world objects to enable software interaction with them.
Detecting real world objects is however a non-trivial task, and thus a wide range of possibilities for accomplishing that exist.

One of these is the usage of so-called markers – visually significant patters – to enable fast and easy detection of objects.
To enable quick and easy development of applications that use marker-based tracking for the augmentation of our reality, this paper and the work it covers was envisioned.

This paper represents the initial design and development work for an Android framework to simplify the detection and usage of markers in Augmented Reality applications.
The framework shall henceforth be named \textit{Imagine} to differentiate it from other similar frameworks and the libraries it utilizes.

Imagine will initially only target applications written for Android using Java \cite{android}.
It will utilize the OpenCV for Android library, a sub-project of the original OpenCV framework specifically targeted for Android devices \cite{opencvandroid}.
By separating the detection and the rendering modules within Imagine, it should be relatively easy to extend the functionality of Imagine to other platforms and other technologies, such as desktop applications for Linux or rendering with DirectX instead of OpenGL ES.

To develop and test the framework, we will also implement an application that relies on Imagine for basic functionality.
The application will allow for the real-time viewing of virtual objects on top of a marker and be used to test and evaluate Imagine.
This shall prove the capability and initial concept of the framework and also serve as a starting point for any future derivative work.

This paper consists of all the work done around the actual implementation.
We will look at the basic requirements of the framework.
Basic necessities will be listed and reviewed.
We'll also present a proposal for the structure of the framework.
Difficulties that arose during development will also be documented, along with possible solutions and commentary.
At the end we will also shortly compare Imagine to other frameworks to put it into perspective.

Apart from the programmatic development of the framework and the final review of our work, we will also deliver at least basic documentation for the framework.
For the framework this will include a Javadoc \cite{docjava} file for the completed code and a basic tutorial for usage.
The implementing application should be self explanatory, although care will be taken to ensure a low learning hurdle for using it.

\section{Project Context}

This paper is the bachelor thesis of Tamino Hartmann, written at the Faculty of Engineering and Computer Science \cite{faculty} at the Institute of Databases and Information Systems at the University of Ulm \cite{ulmuni}, Germany.
The work commissioned is to create a framework for fast and easy integration of marker-based tracking for possible future projects within the institute, thus decreasing repetitive re-implementation of the same features and allowing a faster development time for implementing applications.
The supervisor was Marc Schickler and the examiner was Professor Doctor Manfred Reichert.

\section{Content of this Paper}

Chapter \ref{framework} presents the preliminary work done for finding an algorithm and subsequently for implementing it.
We then shortly take a look at what encompasses our markers for Imagine in chapter \ref{section_markers}.

Chapter \ref{results} presents our encountered difficulties, implemented features, and a comparison to other, similar frameworks.
We conclude with chapter \ref{conclusion}, where we step back and give a more general outlook on possible future work.
