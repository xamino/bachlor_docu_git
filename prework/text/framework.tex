In this section, we take a close look at the proposed structure and capabilities of the framework's API.
The final implementation will however deviate with a high probability.

\subsection{TODO, but include: Marker}

\begin{figure}
	\centering
	\includegraphics[width=4cm]{images/marker_example.png}
	\caption[Example Marker.]{An example of a marker. This image is either printed or displayed by some other means in the real world to allow a system to use it as a reference to base a virtual overlay off of it.}
	\label{fig:marker_example}
\end{figure}

To enable the framework to detect a 3d coordinate system from a video feed, a marker will be required.
A marker is a visually significant pattern that the system can detect within an image and be used to calculate spatial coordinates. Figure \ref{fig:marker_example} shows an example for such a marker.
