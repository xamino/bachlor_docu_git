Finally we will give a broad overview of the accomplishments of this work, difficulties encountered, possible improvements, and use cases.
We will finish with a broad closing statement.

\section{Future Possibilities}
\label{future}

While basically completely usable, there are many aspects that can still be improved.
These range from performance considerations, porting Imagine to other platforms, to extending the functionality offered.
In this section, we will propose some aspects that we think could be of interest for future work.

\subsection{Performance}

Concerning the performance of Imagine a lot could still be done, although globally the performance will remain limited by the capabilities of OpenCV for Android.
One of the main aspects that could be tried however is writing the main method that detects markers using OpenCV with native C++ and calling that from the Java environment with a native call.
We did not try that because of two factors: one, a basic Java framework was the primary goal, and two, developing a Java application for Android with native code in C++ would have put this work beyond our time frame.
It would also have further divided our attention away from the core functionality, as considerable time would have had to be spent learning and utilizing the native development kit.

We are also confident that some aspects of the OpenCV code could be optimized even further, although that might require a more significant analysis with better and robuster timing features.
The performance of the worker threads when using multithreading could certainly also be improved further, as multithreading is hard to do correctly.

Apart from Imagine, improvements in the runtime environment – Android itself – could yield further speed gains, although that would most definitely be beyond the scope of the framework and take a lot of time.

Apart from software aspects various hardware considerations could also lead to better performance.
One would be the increase of the speed at which the camera generates preview images, allowing OpenCV to grab these faster for further accessing.
A higher count of CPU cores could also serve to increase the speed by running more worker threads.
A faster CPU than the 1.4 Ghz quadcore used here will should also increase performance.

\subsection{Features}

While Imagine offers a basic rendering system, it lacks a decent rendering system for production use that can fully utilize the OpenGL ES 2.0 framework.
Shading, texturing, and animation would benefit the use-cases greatly.
These features however would be a project unto itself and was thus left for a future time.

Another feature that would make using Imagine easier is implementing either a lookup model for camera values or offering an internal method for generating them.
This would allow the removal of determining these numbers externally into the framework, further abstracting away possible hurdles to using it.

To improve resistance against marker occlusion and increase marker detection precision, support for marker boards such as used by Aruco could also be added.
However we believe that implementation of such a capability would only be feasible once Imagine has better performance, as the number of markers required for a marker board greatly decrease the framerate.

\subsection{Improvements}
\label{improvements}

From an architectural standpoint, the flow of data within the framework could be improved.
Especially the transfer of data to and from the worker threads could surely be solved better to remove execution blocks.
Most notably the fetching of input frames could be resolved so that the live preview drawn in the background were independent in speed from the worker threads.
This would however require extensive knowledge of multi-threading, and was thus deemed beyond the scope of this project.

The external application programming interface could be expanded too to allow a better and more detailed control of the inner workings.
Care should be taken to keep the basic simplicity of using the framework though.
Internal states that could be opened to external control include the resolution of the images, rendering settings, and marker properties.

Internally, Imagine could be expanded to refine corners using subpixel interpolation when detecting the contours of possible markers.
The OpenCV method for subpixel precision is however expensive in terms of performance.
The advantage of subpixel interpolation would be a better pose estimation, removing sudden jumps and improving the visual fidelity.

\subsection{Extended Possibilities}

Speaking on a more general view, some extended possibilities for future work offers themselves.
One of the more interesting ones would be the porting of the framework to other platforms, most likely for desktop computers first as they offer significantly more processing power.
Higher graphical fidelity and more features for the rendering side would also become a possibility.

A further possibility would be the harnessing of further OpenCV functionality concerning the adaptability of the output.
This could range from situation aware rendering concerning the lighting of models to detection of further information from objects found in the scene.

\section{Closing Statement}

We have shown our results of implementing a Java framework for Android for marker based detection to be used in augmented reality programs.
This includes preparatory work for the framework and the basic application that utilizes it, to the finished results.
The framework, called Imagine, covers the basic required features to be fully functioning, although various aspects could still be greatly improved and expanded upon.

Apart from the framework, documentation and explanation of its workings have been created.
Documentation is to be found in the Javadoc of the project, also seamlessly accessible by any modern IDE during active development.
This paper highlights the internal functionality of Imagine and how to access it.
Insight into debugging capabilities and possible areas where difficulties might arise are also highlighted.

We have also shown some possibilities for further work that would significantly expand the use-cases of our framework.
To help with future evaluations we have also done some basic benchmarks and timed the currently most processing intensive part – the detection algorithm.

All in all, we are satisfied with the promise of Imagine, although in detail there are many areas worthy of more development time.
We consider the framework to be academically usable, but would refrain from utilizing it in a production scenario due to low performance while detecting markers.
We believe the code of Imagine however to be an excellent starting point to learn how to implement augmented reality applications as it provides at the very least a basically functional framework.
